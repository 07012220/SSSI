% $Id: SEGY_Documentation.tex,v 1.1 2004/06/21 21:23:36 cmhogan Exp $
\documentclass[10pt]{article}
\usepackage{xspace}
\title{SEGY manipulation libraries documentation}
\author{Chad M. Hogan}
\date{June 18 2004}

\newcommand{\segy}{{\scriptsize SEGY}\xspace}
\newcommand{\crewes}{{\scriptsize CREWES}\xspace}
\newcommand{\ascii}{{\scriptsize ASCII}\xspace}
\newcommand{\ebcdic}{{\scriptsize EBCDIC}\xspace}
\newcommand{\cdp}{{\scriptsize CDP}\xspace}


\begin{document}
\maketitle

\noindent
This article documents the \segy manipulation libraries found in the
\crewes software collection. Specifically documented are the functions:
\begin{itemize}
\item {\tt SEGY\_GetBinaryHeader}
\item {\tt SEGY\_GetTextHeader}
\item {\tt SEGY\_GetTrace}
\item {\tt SEGY\_ReadBinaryHeader}
\item {\tt SEGY\_ReadTextHeader}
\item {\tt SEGY\_ReadTrace}
\item {\tt SEGY\_WriteBinaryHeader}
\item {\tt SEGY\_WriteGathers}
\item {\tt SEGY\_WriteStack}
\item {\tt SEGY\_WriteTextHeader}
\item {\tt SEGY\_WriteTrace}
\end{itemize}
This set of functions is designed to generate standards-compliant
\segy rev. 1 (May, 2002) files. It includes a full \ebcdic (or \ascii,
if you must) text header, a binary header, and traces with trace
headers and data written in big-endian (or little-endian, if you must)
format.

\subsection*{Warning}
I am not 100\% comfortable with my geometry calculations within the
sample functions {\tt SEGY\_WriteGathers} and {\tt SEGY\_WriteStack}. I
think they're correct, but it is difficult for me to test them
exhaustively. If you are writing \segy and you are confident in your
setting but you think that the produced \segy file is in error with
respect to geometry, it's entirely possible that you are correct.
Please contact me and we'll discuss it.

\section{A Short Introduction to SEGY}

The \segy data format was first defined a long, long time ago. It
arranges things for the convenience of punch-card users, it defaults
to \ebcdic text format, and big-endian data. The intent of this
library is to write a relatively easy-to-use library of \segy writing
utilities. These are meant to be used as tools incorporated into
larger programs, so they are not optimized for use directly from the
command line in order to save a random amount of seismic data to a
file. 

The \segy data format consists of three main parts. The first 3200
bytes in a file is defined as a text header. This text header,
usually in \ebcdic, describes various notes and details that may be
interesting to processing and interpreting staff. Although conventions
are defined, this block is fairly free-form and cannot be reliably
used to automate processing (for example). It is meant to be read by
humans, not computers. This is more of a job for the next part of the
\segy file.

Following the 3200-byte text header is the binary header. The binary
header is 400 bytes of strictly ordered numbers, usually in big-endian
format. These numbers define parameters about the entire data set
contained in the file. The line number, the sampling interval, the
number of samples per trace, the sorting of the traces, and many other
parameters are set in this binary header. These elements are designed
to allow computers to quickly understand the data within the file so
that they can reliably read and process the traces.

The final part of the \segy file contains the traces themselves. Each
trace has two parts: a trace header and the trace data. The trace
header contains information specific to each trace in binary
format. This will contain information such as the trace number, the
source location for the trace, the receiver group location, the time
sample interval, and many others. Following this trace header is the
actual trace series consisting of all the samples one after
another as a string of binary numbers. There is no ``end of trace''
marker, which is why it is important to ensure that the number of
samples is correctly specified in the binary and trace headers.

For a thorough explanation of the \segy data format, please see the
{\scriptsize SEG} website at {\tt http://www.seg.org}.

\section{Quick Start and Examples}
So, you want to just get on with it and write some \segy already?  Ok.
This is your section. The first thing to do is to generate your data.
You'll want it arranged in one of the following ways:
\begin{itemize}
\item In gathers (\cdp or shot). 
\item In a stack.
\end{itemize}

In the case of gathers, then you will have the entire set of gathers
stored in a cell array, with one cell for each gather. Within each
gather, the traces will be arranged in an $N \times M$ array, where
$N$ is the number of samples in each trace, and $M$ is the number of
traces in the gather. For example, we'll imagine that you have all of
your gathers stored in a cell array called {\tt gathers}. The third
gather in this set is:

\begin{verbatim}
thirdgather = gathers{3};
\end{verbatim}

The second trace within this gather is referenced like this:

\begin{verbatim}
secondtrace = thirdgather(:, 2);
\end{verbatim}

\subsection{SEGY\_WriteGathers}

This function is written mostly as an example for you to follow if you
are interested in generating \segy in your own programs. However, it
is directly useful if you are willing to manipulate your data into the
format required by this function.

If you can put your data into that format, then you are entirely set
for writing \segy data. You simply use the {\tt SEGY\_WriteGathers}
function as follows:

\begin{verbatim}
SEGY_WriteGathers(filename, gathers, dt, type, g1, g2, text, num);
\end{verbatim}
Where {\tt filename} is a string containing the full path to the
filename that you want to write, {\tt gathers} is your cell array of
gathers as previously described, {\tt dt} is the sampling interval in
second, {\tt type} is a string containing either 'cdp' or 'shot' to
describe the type of gather, {\tt g1} and {\tt g2} are geometry
parameters that will be described shortly, {\tt text} is the text
header format -- either 'ebcdic' or 'ascii', and {\tt num} is the
number format -- either 'b' for big-endian or 'l' for little endian.

The two geometry parameters {\tt g1} and {\tt g2} require a bit of
explanation. They mean different things for the different kinds of
gathers.
\subsubsection{Writing \cdp gathers}
To write a \cdp gather, you need two elements for the geometry. The
first parameter, {\tt g1}, represents the horizontal distance between
the depth points within a given gather. If your \cdp gathers are each
separated by 100 meters, then {\tt g1} will be 100.

The second parameter, {\tt g2}, represents the offset interval from
the shot point to the receivers within the gathers. So, if the offsets
in a particular \cdp gather are something like: 0, 100, 200, 300,
\ldots; then your interval is 100. Note that this particular function
assumes you have a zero offset shot. Since I imagine that this library
will primarily be used for writing synthetic data, this isn't a bad
thing.

From this, geometry will be calculated. It is assumed that the first
trace in the gather is the zero-offset gather, and the last trace is
the longest-offset gather. Obviously this is going to be
a very limited sort of geometry. If you have more complicated data and
geometry, then you will have to create your own writing function using
the more lower level functions described later.

\subsubsection{Writing shot gathers}
To write a shot gather, you need two elements for the geometry that
are slightly different from the \cdp case. The first parameter, {\tt
  g1}, represents the distance between each shot point. If your shots
are separated by 50 meters, then {\tt g1} will be 50. 

The second parameter, {\tt g2}, represents the offset interval between
receivers. So if you have receivers placed every 100 meters, then {\tt
  g2} will be 100. 

Once again, this functionality is limited. It is assumed that there
will be an odd number of traces in this gather. The first and last
trace are the longest offset traces (on either side of the shot) and
the middle trace is a zero-offset trace. If this does not fit your
needs, then you will have to create your own custom writing routine
using the lower level functions described in the rest of this
document.

\subsubsection{Example}
This function call will generate a \segy file in {\tt /tmp/test.segy}
using the gathers in {\tt cdpgathers}. The gathers are sampled every 2
milliseconds, and they are sorted into \cdp gathers. Each gather's
\cdp is separated by 50 meters, and the shot-to-receiver interval is
100 meters.
\begin{verbatim}
    SEGY_WriteGathers('/tmp/test.segy', cdpgathers, 0.002, ...
                      'cdp', 50, 100, 'ebcdic', 'b');
\end{verbatim}

\subsection{SEGY\_WriteStack}
This is another example function that you may use if it is
convenient. It is much more likely that this will be useful to you for
writing stacks, as stack geometry is a lot simpler than gather
geometry.

It is called as follows:
\begin{verbatim}
SEGY_WriteStack(filename, stack, dt, separation, text, num);
\end{verbatim}

Where {\tt filename} is a string containing the filename to write,
{\tt stack} is an $N \times M$ array of $N$ samples for each of $M$
traces in the stack. {\tt dt} is the sample interval in seconds, {\tt
  separation} is the horizontal separation distance between each
stacked trace, {\tt text} is either 'ebcdic' or 'ascii', and {\tt num}
is either 'b' for big-endian or 'l' for little-endian. 

\subsubsection{Example}

This call will save the stack in {\tt stackdata} into the file {\tt
  /tmp/stacked.segy} with a sample interval of 2 ms. The text will be
  written in \ebcdic and the data in big-endian format. 
\begin{verbatim}
SEGY_WriteStack{'/tmp/stacked.segy', stackdata, 0.002, 100, ...
                'ebcdic', 'b');
\end{verbatim}

\section{Driving Functions}
These functions do all the heavy lifting, when writing \segy files. I
think of them in two separate groups. The first group is the
``retrieving a template'' group:
\begin{itemize}
\item {\tt SEGY\_GetTextHeader}
\item {\tt SEGY\_GetBinaryHeader}
\item {\tt SEGY\_GetTrace}
\end{itemize}
These functions return a template of sorts that can be used as a
starting point for your own binary header, text header, and trace when
you're writing \segy.

The next group is the ``writing to file'' group:
\begin{itemize}
\item {\tt SEGY\_WriteTextHeader}
\item {\tt SEGY\_WriteBinaryHeader}
\item {\tt SEGY\_WriteTrace}
\end{itemize}

\subsection{SEGY\_GetTextHeader}
This function will return a text header that is initialized to be the
correct size as a standard text header for a \segy file. It will be
returned to you in \ascii, but this is ok because {\tt
  SEGY\_WriteTextHeader} expects an \ascii header input as an
argument. All of the \ascii vs. \ebcdic stuff is worked out via
arguments to the writing functions. Don't try to manipulate \ebcdic
directly yourself -- the 1970s are long over.

The text header itself is a 3200 byte string of information. Take a
look at it if you like, and modify it in place to fill in the blanks.
Do {\bf not} change the size of this array. The text header is
generally used purely as notes to the users of the data, so you can
frequently just get a text header as it is, and write it directly
without any changes.
\begin{verbatim}
thead = SEGY_GetTextHeader;
\end{verbatim}

\subsection{SEGY\_GetBinaryHeader}
This function returns an initialized structure that represents all of
the data found in a \segy binary header. The details of the \segy
binary header are documented exhaustively in lots of places. One thing
worth noting is that I tend to name the binary header elements in the
same way as Seismic Unix names them. If you are familiar with Seismic
Unix keywords, then you know all of the keywords for this structure as
well. 

\begin{verbatim}
bhead = SEGY_GetBinaryHeader;
\end{verbatim}

A few of the more important header elements that you probably want to
set in every \segy file are:

\begin{description}
\item[bhead.jobid] job id number
\item[bhead.lino] line number, one line per reel
\item[bhead.reno] reel number
\item[bhead.hdt] sampling interval in microseconds, this reel
\item[bhead.dto] sampling interval in microseconds, in field
\item[bhead.hns] number of samples per trace, this reel
\item[bhead.nso] number of samples, in field
\item[bhead.tsort] trace sorting code
\end{description}

\subsection{SEGY\_GetTrace}
This function returns a template for a \segy trace. As with the other
functions, you get a trace, and then you modify the things that you
want to change. You also have to add in the data for the trace. This
trace structure includes both the trace header and the trace data
itself. So you can use retrieve one template, then for every trace
simply replace the time series data for each new trace, and also
update the relevant geometry and trace id number information.

\begin{verbatim}
trace = SEGY_GetTrace;
\end{verbatim}

Here are a few of the most important trace header elements. Again,
they follow Seismic Unix keyword names.

\begin{description}
\item[trace.tracl] trace sequence number within the line
\item[trace.tracr] trace sequence number within the reel
\item[trace.fldr] field record number
\item[trace.ep] energy source point number
\item[trace.cdp] ensemble number (doesn't have to be cdp)
\item[trace.cdpt] trace number within the ensemble
\item[trace.offset] distance from source to receiver
\item[trace.sx] x source coordinate
\item[trace.gx] x group coordinate
\item[trace.dt] sample interval in microseconds
\item[trace.data] the time series, as an array
\end{description}

Some of these elements will likely be the same for every single trace
within a line. For example, {\tt trace.dt} is probably going to be the
same for every trace. Others must be altered, usually in a non-trivial
way, to accomodate the geometry of the shooting.


\subsection{SEGY\_WriteTextHeader}

This function will write the text header to the \segy file.

\begin{verbatim}
SEGY_WriteTextHeader(FILE, thead, format);
\end{verbatim}

{\tt FILE} is the opened file handle for the \segy file, {\tt thead}
is the text header as returned by {\tt SEGY\_GetTextHeader} and
subsequently modified, {\tt format} is equal to 'ebdic' or 'ascii'.
You should probably choose 'ebcdic' unless you know otherwise, because
\ebcdic conforms to the original \segy standard.

If there is a text header already in the file that you are writing to,
it will be overwritten without warning. You cannot use this function
to insert a text header in front of a binary header and/or traces. It
will simply overwrite the first 3200 bytes of a file without regard
for what exists in those 3200 bytes.

\subsection{SEGY\_WriteBinaryHeader}

This function writes the binary header to the file.

\begin{verbatim}
 SEGY_WriteBinaryHeader(FILE, bhead);
\end{verbatim}

{\tt FILE} is the opened file handle for the \segy file, and {\tt
  bhead} is the binary header as returned by the {\tt
  SEGY\_GetBinaryHeader} function and subsequently modified. Note that
endianness is determined within your call to {\tt fopen}, and is not a
part of this library.

If there is a binary header in the file, this function will overwrite
it without warning. This function will overwrite the bytes from 3201
to 3600 (inclusive) without regard for what exists in those
bytes. Therefore, you cannot use this to insert a binary header into a
file that does not have one. 

\subsection{SEGY\_WriteBinaryHeader}

This function will write a \segy trace to the file. 

\begin{verbatim}
SEGY_WriteTrace(FILE, trace);
\end{verbatim}

{\tt FILE} is the opened file handle for the \segy file and {\tt
  trace} is the trace as returned by the {\tt SEGY\_GetTrace} function
  and subsequently modified. Note that endianness is determined within
  your call to the {\tt fopen}, and is not a part of this library.

The trace will be written to the file at exactly the point where the
current file pointer resides. If you do not understand the previous
sentence, then simply ensure that you include the following line
before you try to use the function {\tt SEGY\_WriteTrace}:

\begin{verbatim}
fseek(FILE, 0, 'eof');
\end{verbatim}

\end{document}

%%% Local Variables: 
%%% mode: latex
%%% TeX-master: t
%%% End: 
